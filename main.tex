\documentclass[a4paper, 12pt]{article}
\usepackage[utf8]{inputenc}
\usepackage[ngerman]{babel}
\usepackage{graphicx}
\usepackage[nottoc]{tocbibind}
\usepackage{algorithm}
\usepackage{algorithmic}
\usepackage{listings}
\usepackage{xcolor}
\usepackage{hyperref}
\usepackage[newfloat]{minted}
\usepackage{url, hyphenat}
\usepackage[top=25mm, left=25mm, right=25mm, bottom=20mm]{geometry}
\usepackage[onehalfspacing]{setspace}
\usepackage{dirtree}
\usepackage{csquotes}

\begin{document}

\begin{titlepage}

\newcommand{\HRule}{\rule{\linewidth}{0.4mm}} % Defines a new command for the horizontal lines, change thickness here

\center % Center everything on the page
 
%----------------------------------------------------------------------------------------
%	HEADING SECTIONS
%----------------------------------------------------------------------------------------

\includegraphics[scale=0.5]{images/hska_logo.jpg}\\[2.5cm] % Include a department/university logo - this will require the graphicx package
\textsc{\Large Seminararbeit}\\[0.5cm] % Major heading such as course name \\
\Large \textsc{Technologiegestütztes Lernen}
%----------------------------------------------------------------------------------------
%	TITLE SECTION
%----------------------------------------------------------------------------------------

\HRule \\[0.4cm]
{ \huge \bfseries Datenschutz und Learning Analytics}\\[0.4cm] % Title of your document
\HRule \\[1.5cm]
 
%----------------------------------------------------------------------------------------
%	AUTHOR SECTION
%----------------------------------------------------------------------------------------

\begin{minipage}{0.4\textwidth}
\begin{flushleft} \large
\emph{Authors:}\\
Niklas \textsc{Kreutzarek}\\ Moritz \textsc{Ulte} \\Claudio \textsc{Busse}\\ % Your name
\emph{Betreuer:}\\
\mbox{Dipl.-Inf. Alexander \textsc{Streicher}} \\
\mbox{Dipl.-Inf. Dipl.-Ing.-Päd. Martin \textsc{Mandausch}} 


\end{flushleft}

\end{minipage}\\[2cm]

% If you don't want a supervisor, uncomment the two lines below and remove the section above
%\Large \emph{Author:}\\
%John \textsc{Smith}\\[3cm] % Your name

%----------------------------------------------------------------------------------------
%	DATE SECTION
%----------------------------------------------------------------------------------------

{\large \today}\\[2cm] % Date, change the \today to a set date if you want to be precise

\vfill % Fill the rest of the page with whitespace

\end{titlepage}

\tableofcontents

\newpage

\section{Inhalt}

Für die automatische Analyse des Lernfortschritts eines NUtzers werden Verfahren aus dem Bereich Learning-Analytics
eingesetzt, also die Analyse von Daten wie Nutzerinteraktionen, Lernprofilen, usw. Die
Datenerfassung erfolgt dabei zielgerichtet, um beispielsweise die Lernumgebung automatisch an die
Bedürfnisse des Nutzers zu adaptieren. Fraglich ist, inwieweit die automatische Datenerfassung mit
geltenden Datenschutzbestimmungen konform gehalten werden können. Sind für eine Adaption
personenbezogene Daten notwendig? Wie können diese automatisch anonymisiert werden? Was
sind technische Lösungen dafür? Wie ist der aktuelle Stand der Forschung und Technik bei der
anonymisierenden Datenerfassung und Benutzerprofilerstellung? Welche positiven und negativen
Beispiele sind bekannt?
Der konzeptionelle Teil der Arbeit zeigt anhand eines Beispiels auf, wie Learning Analytics und
Datenschutz funktionieren kann, unter Einbeziehung automatisch anonymisierender Verfahren. Das
Anwendungsbeispiel sind adaptive digitale Lernspiele für die Bildauswertung

\section{Einleitung}

In dieser Arbeit geht es, um Learning-Analytics mit Bezug auf den Datenschutz. Zuerst werden Grundlagen geklärt, daraufhin wird der aktuelle Stand der Technik und Forschung vorgestellt und zuletzt wird anhand eines Anwendungsbeispiels aufgezeigt, wie mithilfe der zuvor vorgestellten Grundlagen, Learning-Analytics und Datenschutz funktionieren kann.

\subsection{Motivation}

\begin{quote}
    ``Everybody’s talking about Big Data and Learning Analytics, but if you don’t solve privacy first it is going to be killed before it has really started."(Larry Johnson, CEO of the New Media Consortium(NMC). \\
    
    Learning Analytics hat zum Ziel, dass Lernen individuell an die Fähigkeiten und den Wissensstand der Lernenden anzupassen. Dafür werden eine vielzahl an Daten gesammelt und ausgewertet. Auf Grundlage der Auswertung werden Rückschlüsse gezogen, wo zum Beispiel Defizite vorhanden sind und deshalb verschärft auf diesen Bereich der Lernfokus gesetzt werden sollte.
\end{quote}

\newpage   
\subsection{Zielsetzung}

Im Rahmen dieses Dokuments soll eine genauere Einsicht in die aktuelle Datenschutz-Grundverordnung, im Folgenden nur DSGVO genannt und deren Auswirkungen auf die Speicherung und Nutzung gegeben werden. Dazu wird zuerst erklärt was das DSGVO beinhaltet, auf welche Daten und Situationen es sich bezieht. Im Anschluss werden Herausforderungen die sich durch das DSGVO im Bereich Learning-Analytics und Big-Data genauer erläutert und es wird versucht Ansätze und Lösungsvorschläge zu geben, um diese zu bewältigen.

\section{Grundlagen}

In diesem Kapitel werden die Grundlagen des DSGVO sowie nötigen Begrifflichkeiten erläutert. Dazu wird zuerst auf den Datenschutz allgemein und anschließend auf das DSGVO im genaueren eingegangen. Dazu zählt auch was Datenschutz und die DSGVO für Learning-Analytics und Big-Data bedeutet.

\subsection{Datenschutz}

Im Internet oder bei der Nutzung von Software geben Nutzer oft eine Vielzahl von Informationen preis, die von einem Unternehmen gespeichert, verarbeitet und ausgewertet werden könnten. Dabei ist die Informationsmenge über die Jahre stetig gestiegen und auch ihre Auswertung und Nutzung wurde zunehmend komplexer. Auch ist das Bewusstsein von Nutzern über die Preisgabe ihrer persönlichen Daten nicht immer ausgeprägt, sie geben also oft Informationen preis, in der Annahme die Daten seien sicher.\\
Werden viele Daten gesammelt, sei es von Unternehmen oder auch vom Staat, so taucht immer wieder ein Schlagwort auf: ``Der gläserne Mensch"\\
Mit diesem Ausdruck wird die Sorge umschrieben, dass durch wissentlich oder unwissentlich preisgegebene Daten nahezu alles über einen Menschen in Erfahrung gebracht werden kann und die Privatsphäre verschwindet und der Betroffene keine Möglichkeit hat dies zu unterbinden. Zuletzt war dies bei der Debatte um die Vorratsdatenspeicherung präsent.\\

\noindent Die offensichtlichsten Daten sind hierbei zum Beispiel:
\begin{itemize}
\item Vor- und Nachnamen
\item Adresse 
\item Alter und/oder Geburtsdatum
\item Bankdaten
\end{itemize}

\newpage
\noindent 
Einem Unternehmen stehen oft jedoch auch Daten zur Verfügung, von denen ein Nutzer oft nicht weiß, dass er sie einem Unternehmen übermittelt. Dazu gehören unter anderem:
\begin{itemize}
\item Ip-Adresse
\item Browse-Verhalten über Cookies und Tracker
\item Aufrufhäufigkeit einer Website
\item Aufenthaltsort bei Nutzung eines mobilen Gerätes
\end{itemize}

\noindent 
Solche Daten werden allgemein als personenbezogene Daten bezeichnet. In Deutschland gelten solche Daten als personenbezogen, die einer identifizierten oder identifizierbaren natürlichen Person zugeordnet werden können.
Wird im weiteren nur von Daten gesprochen, so handelt es sich dabei immer personenbezogene Daten.\\

\noindent Mittels solcher gesammelten Daten, kann ein Unternehmen viel über seine Nutzer herausfinden. Dies kann dazu genutzt werden personalisierte Angebote zu erstellen oder aber auch an den Kunden angepasste Werbung anzuzeigen. Jedoch könnten diese Daten auch an dritte verkauft werden, welche die Daten für andere Zwecke nutzen ohne, dass ein Kunde oder Nutzer davon Kenntnis hat. 
Es stellen sich beim somit Datenschutz für einen Nutzer einige von Fragen:
\begin{itemize}
	\item Wie sicher sind die Daten gespeichert ?
	\item Wer hat zugriff auf die Daten ?
	\item Was für Daten hat ein Unternehmen ?
\end{itemize}
\noindent Datenschutz ist auch aus einem anderem Grund wichtig, viele Daten bedeuten für jene Unternehmen die sie Besitzen ein gewissen Maß an Macht und natürlich Geld. Wie bereits angesprochen können Unternehmen solche Daten nutzen um Werbung zu personalisieren oder anonyme Daten an andere Unternehmen verkaufen. So verdiente Google durch seine Informationen über Nutzer Ende 2016 etwa 79 Milliarden Dollar durch angepasste Werbung und deren Einnahmen.\\ Ein anderes Beispiel wie über die Sammlung von Daten Geld verdient werden kann ist die Schufa. Eine Anfrage an dieses Institut kostet und ist nur möglich wenn personenbezogene Daten, wie die Bonität eines Betroffenen, vorhanden sind.\\
Grade von Staatlichen Stellen ist ein inzwischen beliebtes Argument um viele Daten sammeln zu können die Terrorvermeidung. So wird versucht Telefondaten, Chat-Verläufe und andere Daten zu speichern um eventuell auf sie zugreifen zu können falls es nötig wird. 

\noindent Durch den Datenschutz soll ein Nutzer vor missbräuchlicher Verarbeitung seiner Daten geschützt werden und sein Recht auf informationelle Selbstbestimmung soll gewährleisten werden. Ebenso dient es als Wahrung seiner Privatsphäre und soll dem Nutzer/Kunden ermöglichen zu entscheiden, wem er wann welche seiner persönlichen Daten zugänglich macht. Wichtig ist hierbei vor allem das Recht auf informationelle Selbstbestimmung, da es unter das in \textbf{Art 1 Abs.1 GG} bestimmte allgemeine Persönlichkeitsrecht fällt. Nach diesem Recht hat jeder Bürger und somit auch Nutzer einer Website oder Services, das Recht über die Preisgabe und Verwendung seiner Daten selbst zu entscheiden.
Somit ist der allgemeine Datenschutz ein Schutz des Kunden bzw. Nutzers vor missbräuchlicher Nutzung seiner Daten und bedeutet, dass die Daten eines Kunden oder Nutzers vertraulich behandelt werden. Dieser Datenschutz bezieht sich jedoch nicht nur auf das Internet sondern auf alle Bereiche in denen Daten eines Kunden oder Nutzers gespeichert werden. Ebenso sollen dem übermäßigen Sammeln von Daten enge Grenzen gesetzt werden.\\

\noindent Die rechtliche Grundlage für den Datenschutz in Deutschland stellt das \textbf{Bundesdatenschutzgesetz(BDSG)} dar, welches im Jahre 2003 in Kraft gesetzt wurde. Es soll die Daten der Bundesbürger vor unbefugten Zugriffen und Missbrauch schützen. Genauer gibt es Einrichtungen, welche Daten von einer Person überlassen bekommen haben vor, diese Daten effektiv zu schützen und nur dann zu erheben und zu sammeln, wenn die Person dem effektiv auch zugestimmt hat. Dazu gehört auch die Möglichkeit die Daten zur Kontrolle jederzeit einsehen zu können. Weiterhin beschränkt das BDSG welche Daten erhoben werden dürfen, wann dies geschehen darf und zu welchem Zweck.


\cite{wikiDatenschutz,datenschutz_internet}



\subsection{DSGVO}

\subsection{Intelligente Lernumgebung}

\subsection{ITS}

\subsection{Digitale Lernspiele (Educational Serious Games)}

\section{Stand der Forschung und Technik}

\subsection{Datenschutz \& E-Learning}

\subsection{Datenschutz/DSGVO \& Learning Analytics}

\subsection{Anonymisierungsmethoden}

\subsection{Pseudonymisierung}

Eine weitere Art des Schutzes der zugänglich gemachten Daten eines Nutzers stellt die sogenannte Pseudonymisierung dar. Im Gegensatz zur vollständigen Anonymisierung, welche im vorgehenden Kapitel erläutert wurde, werden hierbei personenbezogene oder persönliche Daten durch das Ersetzen von Kennzeichen oder Pseudonymen verändert. Dies soll eine Identifikation und Bestimmung des Betroffenen ausschließen oder zumindest deutlich erschweren. Jedoch ist es durch die Zusammenführung der Daten weiterhin möglich auf die Person zu schließen. Wichtig ist hierbei das die Zuordnung der Pseudonyme zu den tatsächlichen Daten separat und gut gesichert gespeichert wird, sodass der Aufwand diese Daten zu erlangen erheblich vergrößert wird. Was in diesem Kontext unter ``erheblich" zu verstehen ist, würde den Rahmen dieser Arbeit übersteigen, da dies eher in die Zuständigkeit eines Juristen bzw. Datenschutzbeauftragen fällt und im Einzelfall geklärt werden muss. Rechtlich fallen Pseudonymisierte Daten weiterhin unter den Datenschutz.
\cite{thesing_anonym_pseudonym, datenschutz_pseudonymisierung}

\noindent Anbei sei noch ein Beispiel für eine Pseudonymisierung gegeben.

Vor der Pseudonymisierung:\\

\begin{tabular}{lcrr}
	
	ID & Vorname & Nachname & Bank \\
	\hline
	0001 & Peter & Meyer & Taunusbank
	
\end{tabular}\\

Nach der Pseudnonymisierung\\

\begin{tabular}{lc}
	
	ID & Bank \\
	\hline
	0001 & Taunusbank
	
\end{tabular}\\\\

\begin{tabular}{lcrr}
	
	ID & Vorname & Nachname\\
	\hline
	0001 & Peter & Meyer
	
\end{tabular}

\section{Konzeption: E-Learning + Serious Gaming + Learning Analytics + Datenschutz/DSGVO}

\section{Szenario: Learning Analytics für Bildauswertung-Lernspiel}

\section{Fazit \& Ausblick}

\newpage
\listoffigures
\bibliographystyle{unsrt}
\newpage
\bibliography{literatur}

\end{document}