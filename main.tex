\documentclass[a4paper, 12pt]{article}
\usepackage[utf8]{inputenc}
\usepackage[ngerman]{babel}
\usepackage{graphicx}
\usepackage[nottoc]{tocbibind}
\usepackage{algorithm}
\usepackage{algorithmic}
\usepackage{listings}
\usepackage{xcolor}
\usepackage{hyperref}
\usepackage[newfloat]{minted}
\usepackage{url, hyphenat}
\usepackage[top=25mm, left=25mm, right=25mm, bottom=20mm]{geometry}
\usepackage[onehalfspacing]{setspace}
\usepackage{dirtree}
\usepackage{csquotes}

\begin{document}

\begin{titlepage}

\newcommand{\HRule}{\rule{\linewidth}{0.4mm}} % Defines a new command for the horizontal lines, change thickness here

\center % Center everything on the page
 
%----------------------------------------------------------------------------------------
%	HEADING SECTIONS
%----------------------------------------------------------------------------------------

\includegraphics[scale=0.5]{images/hska_logo.jpg}\\[2.5cm] % Include a department/university logo - this will require the graphicx package
\textsc{\Large Seminararbeit}\\[0.5cm] % Major heading such as course name \\
\Large \textsc{Technologiegestütztes Lernen}
%----------------------------------------------------------------------------------------
%	TITLE SECTION
%----------------------------------------------------------------------------------------

\HRule \\[0.4cm]
{ \huge \bfseries Datenschutz und Learning Analytics}\\[0.4cm] % Title of your document
\HRule \\[1.5cm]
 
%----------------------------------------------------------------------------------------
%	AUTHOR SECTION
%----------------------------------------------------------------------------------------

\begin{minipage}{0.4\textwidth}
\begin{flushleft} \large
\emph{Authors:}\\
Niklas \textsc{Kreutzarek}\\ Moritz \textsc{Ulte} \\Claudio \textsc{Busse}\\ % Your name
\emph{Betreuer:}\\
\mbox{Dipl.-Inf. Alexander \textsc{Streicher}} \\
\mbox{Dipl.-Inf. Dipl.-Ing.-Päd. Martin \textsc{Mandausch}} 


\end{flushleft}

\end{minipage}\\[2cm]

% If you don't want a supervisor, uncomment the two lines below and remove the section above
%\Large \emph{Author:}\\
%John \textsc{Smith}\\[3cm] % Your name

%----------------------------------------------------------------------------------------
%	DATE SECTION
%----------------------------------------------------------------------------------------

{\large \today}\\[2cm] % Date, change the \today to a set date if you want to be precise

\vfill % Fill the rest of the page with whitespace

\end{titlepage}

\tableofcontents

\newpage

\section{Inhalt}

Für die automatische Analyse des Lernfortschritts eines Nutzers werden Verfahren aus dem Bereich Learning-Analytics
eingesetzt, also die Analyse von Daten wie Nutzerinteraktionen, Lernprofilen, usw. Die
Datenerfassung erfolgt dabei zielgerichtet, um beispielsweise die Lernumgebung automatisch an die
Bedürfnisse des Nutzers zu adaptieren. Fraglich ist, inwieweit die automatische Datenerfassung mit
geltenden Datenschutzbestimmungen konform gehalten werden können. Sind für eine Adaption
personenbezogene Daten notwendig? Wie können diese automatisch anonymisiert werden? Was
sind technische Lösungen dafür? Wie ist der aktuelle Stand der Forschung und Technik bei der
anonymisierenden Datenerfassung und Benutzerprofilerstellung? Welche positiven und negativen
Beispiele sind bekannt?
Der konzeptionelle Teil der Arbeit zeigt anhand eines Beispiels auf, wie Learning Analytics und
Datenschutz funktionieren kann, unter Einbeziehung automatisch anonymisierender Verfahren. Das
Anwendungsbeispiel sind adaptive digitale Lernspiele für die Bildauswertung

\section{Einleitung}

In dieser Arbeit geht es, um Learning-Analytics mit Bezug auf den Datenschutz. Zuerst werden Grundlagen geklärt, daraufhin wird der aktuelle Stand der Technik und Forschung vorgestellt und zuletzt wird anhand eines Anwendungsbeispiels aufgezeigt, wie mithilfe der zuvor vorgestellten Grundlagen, Learning-Analytics und Datenschutz funktionieren kann.

\subsection{Motivation}

\begin{quote}
    ``Everybody’s talking about Big Data and Learning Analytics, but if you don’t solve privacy first it is going to be killed before it has really started."(Larry Johnson, CEO of the New Media Consortium(NMC). \\
    
    Learning Analytics hat zum Ziel, dass Lernen individuell an die Fähigkeiten und den Wissensstand der Lernenden anzupassen. Dafür werden eine vielzahl an Daten gesammelt und ausgewertet. Auf Grundlage der Auswertung werden Rückschlüsse gezogen, wo zum Beispiel Defizite vorhanden sind und deshalb verschärft auf diesen Bereich der Lernfokus gesetzt werden sollte.
\end{quote}

\newpage   
\subsection{Zielsetzung}

Im Rahmen dieses Dokuments soll eine genauere Einsicht in die aktuelle Datenschutz-Grundverordnung, im Folgenden nur DSGVO genannt und deren Auswirkungen auf die Speicherung und Nutzung gegeben werden. Dazu wird zuerst erklärt was das DSGVO beinhaltet, auf welche Daten und Situationen es sich bezieht. Im Anschluss werden Herausforderungen die sich durch das DSGVO im Bereich Learning-Analytics und Big-Data genauer erläutert und es wird versucht Ansätze und Lösungsvorschläge zu geben, um diese zu bewältigen.

\section{Grundlagen}

In diesem Kapitel werden die Grundlagen des DSGVO sowie nötigen Begrifflichkeiten erläutert. Dazu wird zuerst auf den Datenschutz allgemein und anschließend auf das DSGVO im genaueren eingegangen. Dazu zählt auch was Datenschutz und die DSGVO für Learning-Analytics und Big-Data bedeutet.

\subsection{Datenschutz}

Im Internet oder bei der Nutzung von Software geben Nutzer oft eine Vielzahl von Informationen preis, die von einem Unternehmen gespeichert, verarbeitet und ausgewertet werden könnten. Dabei ist die Informationsmenge über die Jahre stetig gestiegen und auch ihre Auswertung und Nutzung wurde zunehmend komplexer. Auch ist das Bewusstsein von Nutzern über die Preisgabe ihrer persönlichen Daten nicht immer ausgeprägt, sie geben also oft Informationen preis, in der Annahme die Daten seien sicher.\\
Werden viele Daten gesammelt, sei es von Unternehmen oder auch vom Staat, so taucht immer wieder ein Schlagwort auf: ``Der gläserne Mensch"\\
Mit diesem Ausdruck wird die Sorge umschrieben, dass durch wissentlich oder unwissentlich preisgegebene Daten nahezu alles über einen Menschen in Erfahrung gebracht werden kann und die Privatsphäre verschwindet und der Betroffene keine Möglichkeit hat dies zu unterbinden. Zuletzt war dies bei der Debatte um die Vorratsdatenspeicherung präsent.\\

\noindent Die offensichtlichsten Daten sind hierbei zum Beispiel:
\begin{itemize}
\item Vor- und Nachnamen
\item Adresse 
\item Alter und/oder Geburtsdatum
\item Bankdaten
\end{itemize}

\newpage
\noindent 
Einem Unternehmen stehen oft jedoch auch Daten zur Verfügung, von denen ein Nutzer oft nicht weiß, dass er sie einem Unternehmen übermittelt. Dazu gehören unter anderem:
\begin{itemize}
\item Ip-Adresse
\item Browse-Verhalten über Cookies und Tracker
\item Aufrufhäufigkeit einer Website
\item Aufenthaltsort bei Nutzung eines mobilen Gerätes
\end{itemize}

\noindent 
Solche Daten werden allgemein als personenbezogene Daten bezeichnet. In Deutschland gelten jene Daten als personenbezogen, die einer identifizierten oder identifizierbaren, also einer bestimmten natürlichen Person zugeordnet werden können. Diese Daten können somit die betroffene Person identifizieren oder identifizierbar machen. Weiterhin werden bestimmte personenbezogene Daten verschärft geschützt. Zu diesen Daten zählen unter anderem:
\begin{itemize}
	\item Ethnische Herkunft
	\item Politische Meinungen
	\item Religiöse Überzeugungen und Angehörigkeiten
	\item Gewerkschaftszugehörigkeit
	\item Gesundheit und Sexualität
\end{itemize}
\noindent Wird im weiteren nur von Daten gesprochen, so handelt es sich dabei immer personenbezogene Daten. \\

\noindent Mittels solcher gesammelten Daten, kann ein Unternehmen viel über seine Nutzer herausfinden. Dies kann dazu genutzt werden personalisierte Angebote zu erstellen oder aber auch an den Kunden angepasste Werbung anzuzeigen. Jedoch könnten diese Daten auch an dritte verkauft werden, welche die Daten für andere Zwecke nutzen ohne, dass ein Kunde oder Nutzer davon Kenntnis hat. Wird von der Verarbeitung von Daten gesprochen, so fällt darunter jegliche Art von Vorgang bei dem personenbezogene Daten involviert sind.
Es stellen sich beim somit Datenschutz für einen Nutzer einige von Fragen:
\begin{itemize}
	\item Wie sicher sind die Daten gespeichert
	\item Wer hat zugriff auf die Daten
	\item Was für Daten hat ein Unternehmen
\end{itemize}
\noindent Datenschutz ist auch aus einem anderem Grund wichtig, viele Daten bedeuten für jene Unternehmen die sie Besitzen ein gewissen Maß an Macht und natürlich Geld. Wie bereits angesprochen können Unternehmen solche Daten nutzen um Werbung zu personalisieren oder anonyme Daten an andere Unternehmen verkaufen. So verdiente Google durch seine Informationen über Nutzer Ende 2016 etwa 79 Milliarden Dollar durch angepasste Werbung und deren Einnahmen.\\ Ein anderes Beispiel wie über die Sammlung von Daten Geld verdient werden kann ist die Schufa. Eine Anfrage an dieses Institut kostet und ist nur möglich wenn personenbezogene Daten, wie die Bonität eines Betroffenen, vorhanden sind.\\
Grade von Staatlichen Stellen ist ein inzwischen beliebtes Argument um viele Daten sammeln zu können die Terrorvermeidung. So wird versucht Telefondaten, Chat-Verläufe und andere Daten zu speichern um eventuell auf sie zugreifen zu können falls es nötig wird. 

\noindent Durch den Datenschutz soll ein Nutzer vor missbräuchlicher Verarbeitung seiner Daten geschützt werden und sein Recht auf informationelle Selbstbestimmung soll gewährleisten werden. Ebenso dient es als Wahrung seiner Privatsphäre und soll dem Nutzer/Kunden ermöglichen zu entscheiden, wem er wann welche seiner persönlichen Daten zugänglich macht. Wichtig ist hierbei vor allem das Recht auf informationelle Selbstbestimmung, da es unter das in \textbf{Art. 1 Abs. 1 lit. a GG} bestimmte allgemeine Persönlichkeitsrecht fällt. Nach diesem Recht hat jeder Bürger und somit auch Nutzer einer Website oder Services, das Recht über die Preisgabe und Verwendung seiner Daten selbst zu entscheiden.
Somit ist der allgemeine Datenschutz ein Schutz des Kunden bzw. Nutzers vor missbräuchlicher Nutzung seiner Daten und bedeutet, dass die Daten eines Kunden oder Nutzers vertraulich behandelt werden. Dieser Datenschutz bezieht sich jedoch nicht nur auf das Internet sondern auf alle Bereiche in denen Daten eines Kunden oder Nutzers gespeichert werden. Ebenso sollen dem übermäßigen Sammeln von Daten enge Grenzen gesetzt werden.\\
\cite{wikiDatenschutz,datenschutz_internet}

\noindent Die rechtliche Grundlage für den Datenschutz in Deutschland stellt das \textbf{Bundesdatenschutzgesetz(BDSG)} dar, welches im Jahre 2003 in Kraft gesetzt wurde. Es soll die Daten der Bundesbürger vor unbefugten Zugriffen und Missbrauch schützen. Genauer gibt es Einrichtungen, welche Daten von einer Person überlassen bekommen haben, vor diese Daten effektiv zu schützen und nur dann zu erheben und zu sammeln, wenn die Person dem effektiv auch zugestimmt hat. Dazu gehört auch die Möglichkeit die Daten zur Kontrolle jederzeit einsehen zu können. Weiterhin beschränkt das BDSG welche Daten erhoben werden dürfen, wann dies geschehen darf und zu welchem Zweck und legt auch fest, welche Sicherheitsvorkehrungen die Einrichtungen und Unternehmen treffen müssen um die Daten sicher zu hinterlegen.\\Da dieses Gesetz recht umfangreich ist und nicht direkt Thema dieser Ausarbeitung ist wird daher auf den online verfügbaren Gesetzestext verwiesen, welcher auch die fälligen Bußgelder bei Verstößen beinhaltet.
\cite{Bundesdatenschutzgesetz}

\subsubsection{Datenschutz in Unternehmen}

Bisher wurde der Datenschutz nur allgemein betrachtet, weswegen an dieser Stelle kurz auf den Datenschutz innerhalb eines Unternehmens und dessen Auswirkungen auf ein solches eingegangen werden soll.
\\Das BDSG regelt nicht nur den Datenschutz für öffentliche Stellen sondern auch für nicht öffentliche Unternehmen und öffentlich-rechtliche Wettbewerbsunternehmen. Für solche Unternehmen ist das BDSG verbindlich sobald sie personenbezogene Daten nutzen, erheben oder verarbeiten. Dabei betrifft es vor allem jene Daten, welche automatisiert erhoben werden, oder aus automatisierten Verfahren innerhalb des Unternehmens stammen.
Das BDSG begrenzt außerdem die Nutzung personenbezogener Daten auf(§1 Abs. 1 lit. a BDSG):\\
\begin{itemize}
	\item Rechtsgeschäftliche Vorgänge oder rechts-geschäftsähnliche Schuldverhältnisse, für die derlei Daten notwendig sind
	\item Ein vorliegendes berechtigtes Interesse der Stelle, soweit dieses dem schutzwürdigen Interesses des Betroffenen nicht entgegensteht
	\item Daten, die allgemein zugänglich sind
\end{itemize}
\noindent Weiterhin ist für den Datenschutz, der Datenerhebung und deren Verarbeitung eine Zweckgebundenheit zwingend erforderlich. Dadurch soll verhindert werden, dass Unternehmen Daten auf Vorrat sammeln.\\
Die wichtigsten Punkte für Unternehmen sind hierbei:
\begin{itemize}
	\item Der Zugriff und ein möglicher Datenmissbrauch muss mit allen zur Verfügung stehenden Mitteln verhindert werden
	\item Die Nutzung, Erhebung und Verarbeitung zu Werbezwecken, Adresshandel oder Marketing ist nur zulässig, sofern der Betroffene dieser Zweckbindung zustimmt
	\item Es bedarf einer Einwilligung des Betroffenen bei der Erhebung und Verarbeitung einer Daten
	\item Alle Unternehmen sind verpflichtet einen Datenschutzbeauftragten zu ernennen.
	\item Unternehmen dürfen einen erfolgreichen Vertragsabschluss nicht in Abhängigkeit der Einwilligung des Betroffenen stellen, somit gilt das Kopplungsverbot
	\item Anonymisierte Daten müssen getrennt von Daten gehalten werden, welche eine Identifizierung einer Person möglich machen
	\item Ein Unternehmen unterliegt der Auskunftspflicht gegenüber Betroffenen, deren Daten sie erheben, verarbeiten oder nutzen.
	\item Verjährte oder falsche Daten müssen nicht öffentliche Stellen löschen, berichtigen oder zugangssicher speichern
	\item Alle Mitarbeiter eines Unternehmens müssen auf das Datengeheimnis nach §5 BDSG verpflichtet werden, sofern sie mit personenbezogenen Daten arbeiten
\end{itemize}
Abließend ist zu sagen, das der Datenschutz in einem Unternehmen nicht nur für den Kunden wichtig ist, sondern auch für die eigenen Mitarbeiter. Bei Missachtung können hohe Bußgelder drohen, außerdem stärkt ein striktes einhalten der Vorschriften das Vertrauen eines Kunden oder der Mitarbeiter zu einem Unternehmen.
\cite{Bundesdatenschutzgesetz,datenschutz_unternehmen}

\subsection{DSGVO}

Die Datenschutzgrundverordnung, im weiteren nur DSGVO genannt, trat am 25.5.2018 in der gesamten Europäischen Union in kraft und sorgt für eine einheitliche Regelung zum Schutz von personenbezogenen Daten. Es ersetzt somit das BDSG als geltendes Recht. Das DSGVO geht dabei nach dem Prinzip \textbf{Verbot mit Erlaubnisvorbehalt} vor. Dies besagt, dass personenbezogene Daten grundsätzlich nicht Verarbeitet werden dürfen, bis eine ausdrückliche Erlaubnis vorliegt. Es ist dabei irrelevant ob es sich um die manuelle Verarbeitung oder automatisierte Verarbeitung von Daten handelt.\\Diese nötige Erlaubnis kann über zwei Wege erteilt werden:
\begin{itemize}
	\item Gesetzliche Regelung
	\item Einwilligung des Betroffenen
\end{itemize}
\cite{datenschutz_dsgvo}\\
\noindent Das DSGVO bietet hierbei jedoch auch Ausnahmen in der Sachlichen Anwendung, welche in \textbf{Art. 2 lit. a DSGVO} zu finden sind. Darunter fallen unter anderem folgende Tätigkeiten oder Verarbeitungsprozesse:
\begin{itemize}
	\item Tätigkeiten die nicht in den Anwendungsbereich des Unionsrechts fallen
	\item Wenn die Verarbeitung ausschließlich durch natürliche Personen ausgeführt wird und zur Ausübung persönlicher oder familiärer Tätigkeiten dient
	\item Prozesse von zuständigen Behörden zum Zwecke der Verhütung, Ermittlung, Aufdeckung oder Verfolgung von Straftaten oder Strafvollstreckung, sowie des Schutzes vor und der Abwehr von Gefahren für die öffentliche Sicherheit
\end{itemize}
\cite{noauthor_datenschutz-grundverordnung_nodate}\\
Der genaue Räumliche Anwendungsbereich des DSGVO ist genau in \textbf{Art. 3 lit. a DSGVO} festgehalten, allgemein kann jedoch gesagt werden, das es Anwendung findet, sobald personenbezogene Daten eines Betroffenen, welcher sich innerhalb der Union befindet, verarbeitet werden. Ebenso wird es angewendet wenn sich die Verarbeitende Stelle innerhalb der Union befindet.\\Da das DSGVO recht umfangreich ist und die genaue Auslegung durch einen Datenschutzbeauftragten unter Hilfenahme eines Juristen erfolgen sollte, wird weiterhin nur ein grober Überblick über das DSGVO und die Grundsätze der Verarbeitung von personenbezogener Daten gegeben.
\\Das DSGVO sagt über die Verarbeitung von personenbezogenen Daten, dass sie auf rechtmäßige Weiße, nach Treu und Glauben in einer für die betroffene Person nachvollziehbaren Weise verarbeitet werden müssen. Weiterhin müssen diese Daten für einen festgelegten, eindeutigen und legitimen Zweck erhoben werden und dürfen nicht in einer mit diesem Zweck unvereinbaren Weise weiterverarbeitet werden. Eine Ausnahme ist hierbei die Archivierung von Daten für wissenschaftliche oder historische Forschungszwecke, sowie für statistische Zwecke. Auch müssen die Daten auf eine für den Zweck angemessene und minimale Menge beschränkt werden und sachlich richtig und erforderlichenfalls auf dem neuesten Stand sein. Ebenso müssen die Daten in einer Form gespeichert werden, welche die Identifizierung der betroffenen Person nur so lange ermöglicht, wie es für die Zwecke, für die sie erhoben und verarbeitet werden, erforderlich ist. Das heißt, dass personenbezogene Daten nur dann länger gespeichert werden dürfen, genaueres dazu in \textbf{Art. 5 lit. a DSGVO}. Als letzten wichtigen Punkt ist hierbei auch die angemessene Sicherheit aufgeführt, was bedeutet, das die erhobenen Daten vor unbefugter oder unrechtmäßiger Verarbeitung geschützt werden müssen, was geeignete technische und organisatorische Maßnahmen einschließt.
\\Auch die Rechtmäßigkeit der Verarbeitung von personenbezogener Daten wird im DSGVO geregelt. Dazu muss mindestens einer der Bedingungen in \textbf{Art. 6 lit. a DSGVO} erfüllt sein. Beispielhaft seien hier folgende aufgeführt:
\begin{itemize}
	\item Die betroffene Person hat ihre Einwilligung zu der Verarbeitung der sie betreffenden personenbezogenen Daten für einen oder mehrere bestimmte Zwecke gegeben
	\item Die Verarbeitung ist zur Erfüllung einer rechtlichen Verpflichtung erforderlich, der der Verantwortliche unterliegt
	\item Die Verarbeitung ist für die Erfüllung eines Vertrags, dessen Vertragspartei die betroffene Person ist, oder zur Durchführung vorvertraglicher Maßnahmen erforderlich, die auf Anfrage der betroffenen Person erfolgen
	\item Die Verarbeitung ist erforderlich, um lebenswichtige Interessen der betroffenen Person oder einer anderen natürlichen Person zu schützen
\end{itemize}
\cite{noauthor_datenschutz-grundverordnung_nodate}\\
Als Bedingungen für eine Einwilligung gelten laut DSGVO folgende im Auszug von \textbf{Art.7 DSGVO} genannten Punkte:
\begin{itemize}
	\item Beruht die Verarbeitung auf einer Einwilligung, muss der Verantwortliche nachweisen können, dass die betroffene Person in die Verarbeitung ihrer personenbezogenen Daten eingewilligt hat
	\item Erfolgt die Einwilligung der betroffenen Person durch eine schriftliche Erklärung, die noch andere Sachverhalte betrifft, so muss das Ersuchen um Einwilligung in verständlicher und leicht zugänglicher Form in einer klaren und einfachen Sprache so erfolgen, dass es von den anderen Sachverhalten klar zu unterscheiden ist. Teile der Erklärung sind dann nicht verbindlich, wenn sie einen Verstoß gegen diese Verordnung darstellen.
\end{itemize}
\cite{noauthor_datenschutz-grundverordnung_nodate}\\
Wie durch diese Auszüge zu erkennen ist, regelt das DSGVO relativ genau wie mit personenbezogenen Daten umzugehen ist. Die Anwendung ist für die allgemeinen Fälle dabei ebenfalls vorgegeben, bei speziellen Anwendungsgebieten ist es jedoch ratsam einen Juristen zu konsultieren um sich Rechtlich absichern zu können.\\
Auf die weiteren genauen Bestimmungen, Bedingungen und Inhalte des DSGVO kann im Rahmen dieser Arbeit nicht eingegangen werden, da diese, wie bereits vorhergehend erläutert, einerseits sehr umfangreich sind, andererseits unter Zuhilfenahme eines Juristen und Datenschutzbeauftragten durchgegangen werden sollten.

\cite{datenschutz_dsgvo,noauthor_datenschutz-grundverordnung_nodate}

\subsection{Allgemeine Herausforderungen im Bezug auf das DSGVO}
Wird das DSGVO in einem Unternehmen angewendet, so kommt es zu oft zu einer Vielzahl von Herausforderungen die gemeistert werden müssen. In diesem Abschnitt soll auf ein paar der Häufigsten eingegangen werden.



\subsection{Intelligente Lernumgebung}
\begin{quote}
	„Everything will learn. These innovations are beginning to emerge enabled by cloud computing, big data analytics and learning technologies all coming together.“ - IBM
\end{quote}
\noindent Eine Intelligente Lernumgebung passt sich den individuellen Bedürfnissen der Lernenden an. \\
Durch die Sammlung und Auswertung von Daten eines Lernenden und der Daten über die erbrachten Leistungen wird die Lernmethodik und der Inhalt so angepasst, dass der Lernprozess möglichst passend für den Lernenden geschneidert ist. 

\subsection{Intelligentes Tutorsystem}
Ein intelligentes Tutorsystem (ITS) ist ein Computersystem, das darauf abzielt, den Lernenden sofortige und maßgeschneiderte Anweisungen oder Rückmeldungen zu geben. Das Ziel solcher Tutorsysteme ist ein sinnvolles und effektives Lernen zu ermöglichen. Ein Beispiel für so ein System wäre ILIAS, ein Forum in dem es durch verschiedene Plugins ermöglicht wird eine Rückmeldung zum Lernverlauf zu geben.

\subsection{Digitale Lernspiele (Educational Serious Games)}

\section{Stand der Forschung und Technik}

\subsection{Datenschutz \& E-Learning}

\subsection{Datenschutz/DSGVO \& Learning Analytics}

\subsection{Anonymisierungsmethoden}

\subsection{Pseudonymisierung}

Eine weitere Art des Schutzes der zugänglich gemachten Daten eines Nutzers stellt die sogenannte Pseudonymisierung dar. Im Gegensatz zur vollständigen Anonymisierung, welche im vorgehenden Kapitel erläutert wurde, werden hierbei personenbezogene oder persönliche Daten durch das Ersetzen von Kennzeichen oder Pseudonymen verändert. Genauer sind dies solche Daten und Merkmale, welche zur Identifikation einer Person genutzt werden können, also Identifikationsmerkmale. Dies soll eine Identifikation und Bestimmung des Betroffenen ausschließen oder zumindest deutlich erschweren. Jedoch ist es durch die Zusammenführung der Daten weiterhin möglich auf die Person zu schließen. Wichtig ist hierbei das die Zuordnung der Pseudonyme zu den tatsächlichen Daten separat und gut gesichert gespeichert wird, sodass der Aufwand diese Daten zu erlangen erheblich vergrößert wird. Was in diesem Kontext unter ``erheblich" zu verstehen ist, würde den Rahmen dieser Arbeit übersteigen, da dies eher in die Zuständigkeit eines Juristen bzw. Datenschutzbeauftragen fällt und im Einzelfall geklärt werden muss. Rechtlich fallen pseudonymisierte Daten weiterhin unter den Datenschutz und geltende Gesetze diesbezüglich.
\cite{thesing_anonym_pseudonym, datenschutz_pseudonymisierung}

\noindent Anbei sei noch ein Beispiel für eine Pseudonymisierung gegeben:\\

\noindent Vor der Pseudonymisierung:\\

\begin{tabular}{lcrr}
	
	ID & Vorname & Nachname & Bank \\
	\hline
	0001 & Peter & Meyer & Taunusbank
	
\end{tabular}\\

\noindent Nach der Pseudonymisierung\\

\begin{tabular}{lc}
	
	ID & Bank \\
	\hline
	0001 & Taunusbank
	
\end{tabular}\\\\

\begin{tabular}{lcrr}
	
	ID & Vorname & Nachname\\
	\hline
	0001 & Peter & Meyer
	
\end{tabular}

\section{Konzeption: E-Learning + Serious Gaming + Learning Analytics + Datenschutz/DSGVO}

\section{Szenario: Learning Analytics für Bildauswertung-Lernspiel}
SAR-Tutor ist eine Lernsoftware für die Radarbildauswertung. Grundlegende Prinzipien der Radartechnik und insbesondere des synthetischen Apertur Radar (SAR) werden als interaktiver Kurs mit Bilderkennungsaufgaben sowie Wissensfragen beigebracht. Die Ausbildung zur Radarbildauswertung wird in vielen Bereichen wie Küsten- und Gewässerschutz, Umweltschutz, Aufklärung und Überwachung benötigt. \\
\cite{sarTutor}
\section{Fazit \& Ausblick}
Datenschutz, das DSGVO, deren Umsetzung und Einhaltung dieser Bestimmungen sind ein umfangreiches und komplexes Thema. Es gibt viele Bereiche und Punkte, wie die Dokumentation, Weitergabe der Daten an Dritte, Art der Speicherung und Löschung welche beachtet werden müssen und bei denen Unternehmen oft juristischen Rat einholen sollten. Das DSGVO ist so ausgelegt, dass es allgemein gültig ist, also nicht nur für Software, sondern für alle Arten der Verarbeitung von personenbezogenen Daten. Die genaue Umsetzung in einem Unternehmen hängt zudem von den verarbeiteten Daten und der Art der Verarbeitung ab, weshalb in dieser Arbeit darauf nur sehr beschränkt anhand von kleinen Beispielen eingegangen werden konnte. Somit wird eher ein allgemeiner Überblick über den Datenschutz mit einigen Hinweisen und Herausforderungen die sich im Bereich Learning Analytics ergeben können.\\Dennoch ist der Datenschutz ein anhaltendes und wichtiges Thema und bei korrekter Umsetzung schützt es ein Unternehmen nicht nur vor rechtlichen Konsequenzen, es stärkt auch das Vertrauen der Kunden/Nutzer in ein Unternehmen.
\newpage
\listoffigures
\bibliographystyle{unsrt}
\newpage
\bibliography{literatur}

\end{document}