\documentclass[a4paper, 12pt]{article}
\usepackage[utf8]{inputenc}
\usepackage[ngerman]{babel}
\usepackage{graphicx}
\usepackage[nottoc]{tocbibind}
\usepackage{algorithm}
\usepackage{algorithmic}
\usepackage{listings}
\usepackage{xcolor}
\usepackage{hyperref}
\usepackage[newfloat]{minted}
\usepackage{url, hyphenat}
\usepackage[top=25mm, left=25mm, right=25mm, bottom=20mm]{geometry}
\usepackage[onehalfspacing]{setspace}
\usepackage{dirtree}
\usepackage{csquotes}

\begin{document}

\begin{titlepage}

\newcommand{\HRule}{\rule{\linewidth}{0.4mm}} % Defines a new command for the horizontal lines, change thickness here

\center % Center everything on the page
 
%----------------------------------------------------------------------------------------
%	HEADING SECTIONS
%----------------------------------------------------------------------------------------

\includegraphics[scale=0.5]{images/hska_logo.jpg}\\[2.5cm] % Include a department/university logo - this will require the graphicx package
\textsc{\Large Seminararbeit}\\[0.5cm] % Major heading such as course name \\
\Large \textsc{Technologiegestütztes Lernen}
%----------------------------------------------------------------------------------------
%	TITLE SECTION
%----------------------------------------------------------------------------------------

\HRule \\[0.4cm]
{ \huge \bfseries Datenschutz und Learning Analytics}\\[0.4cm] % Title of your document
\HRule \\[1.5cm]
 
%----------------------------------------------------------------------------------------
%	AUTHOR SECTION
%----------------------------------------------------------------------------------------

\begin{minipage}{0.4\textwidth}
\begin{flushleft} \large
\emph{Authors:}\\
Niklas \textsc{Kreutzarek}\\ Moritz \textsc{Ulte} \\Claudio \textsc{Busse}\\ % Your name
\emph{Betreuer:}\\
\mbox{Dipl.-Inf. Alexander \textsc{Streicher}} \\
\mbox{Dipl.-Inf. Dipl.-Ing.-Päd. Martin \textsc{Mandausch}} 


\end{flushleft}

\end{minipage}\\[2cm]

% If you don't want a supervisor, uncomment the two lines below and remove the section above
%\Large \emph{Author:}\\
%John \textsc{Smith}\\[3cm] % Your name

%----------------------------------------------------------------------------------------
%	DATE SECTION
%----------------------------------------------------------------------------------------

{\large \today}\\[2cm] % Date, change the \today to a set date if you want to be precise

\vfill % Fill the rest of the page with whitespace

\end{titlepage}

\tableofcontents

\newpage

\section{Inhalt}

Für die automatische Analyse von Lernerzuständen werden Verfahren aus dem Bereich Learning
Analytics eingesetzt, also die Analyse von Nutzerinteraktionsdaten, Lernerprofilen, usw. Die
Datenerfassung erfolgt dabei zielgerichtet, um beispielsweise die Lernumgebung automatisch an die
Lernerbedürfnisse zu adaptieren. Fraglich ist, inwieweit die automatische Datenerfassung mit
geltenden Datenschutzbestimmungen konform gehalten werden können. Sind für eine Adaption
personenbezogene Daten notwendig? Wie können diese automatisch anonymisiert werden? Was
sind technische Lösungen dafür? Wie ist der aktuelle Stand der Forschung und Technik bei der
anonymisierenden Datenerfassung und Benutzerprofilerstellung? Welche positiven und negativen
Beispiele sind bekannt?
Der konzeptionelle Teil der Arbeit zeigt anhand eines Beispiels auf, wie Learning Analytics und
Datenschutz funktionieren kann, unter Einbeziehung automatisch anonymisierender Verfahren. Das
Anwendungsbeispiel sind adaptive digitale Lernspiele für die Bildauswertung

\section{Einleitung}

In dieser Arbeit geht es, um Learning Analytics mit Bezug auf den Datenschutz. Zuerst werden Grundlagen geklärt, daraufhin wird der aktuelle Stand der Technik und Forschung vorgestellt und zuletzt wird anhand eines Anwendungsbeispiels aufgezeigt, wie mithilfe der zuvor vorgestellten Grundlagen, Learning Analytics und Datenschutz funktionieren kann.

\subsection{Motivation}

\begin{quote}
    ``Everybody’s talking about Big Data and Learning Analytics, but if you don’t solve privacy first it is going to be killed before it has really started."(Larry Johnson, CEO of the New Media Consortium(NMC). \\
    
    Learning Analytics hat zum Ziel, dass Lernen individuell an die Fähigkeiten und den Wissensstand der Lernenden anzupassen. Dafür werden eine vielzahl an Daten gesammelt und ausgewertet. Auf Grundlage der Auswertung werden Rückschlüsse gezogen, wo zum Beispiel Defizite vorhanden sind und deshalb verschärft auf diesen Bereich der Lernfokus gesetzt werden sollte.
\end{quote}

\newpage   
\subsection{Zielsetzung}

Im Rahmen dieses Dokuments soll eine genauere Einsicht in die aktuelle Datenschutz-Grundverordnung, im Folgenden nur DSGVO genannt und deren Auswirkungen auf die Speicherung und Nutzung gegeben werden. Dazu wird zuerst erklärt was das DSGVO beinhaltet, auf welche Daten und Situationen es sich bezieht. Im Anschluss werden Probleme die sich durch das DSGVO im Bereich Learning-Analytics genauer erläutert und es wird versucht Lösungsvorschläge für diese Probleme aufzuzeigen.

\section{Grundlagen}

In diesem Kapitel werden die Grundlagen des DSGVO sowie nötigen Begrifflichkeiten erläutert. Dazu wird zuerst auf den Datenschutz allgemein und anschließend auf das DSGVO im genaueren eingegangen.

\subsection{Datenschutz}


Im Internet oder bei der Nutzung von Software geben Nutzer oft eine Vielzahl von Informationen preis, die von einem Unternehmen gespeichert, verarbeitet und ausgewertet werden könnten.
Die offensichtlichsten Daten sind hierbei zum Beispiel:
\begin{itemize}
\item Vor- und Nachnamen
\item Adresse 
\item Alter und/oder Geburtsdatum
\end{itemize}

\noindent 
Einem Unternehmen stehen oft jedoch auch Daten zur Verfügung, von denen ein Nutzer oft nicht weiß, dass er sie einem Unternehmen übermittelt. Dazu gehören unter anderem:
\begin{itemize}
\item Ip-Adresse
\item Browse-Verhalten über Cookies und Tracker
\item Aufrufhäufigkeit einer Website
\end{itemize}

\noindent 
Mittels solcher gesammelten Daten, kann ein Unternehmen viel über seine Nutzer herausfinden. Dies kann dazu genutzt werden personalisierte Angebote zu erstellen oder aber auch an den Kunden angepasste Werbung anzuzeigen. Jedoch könnten diese Daten auch an dritte verkauft werden, welche die Daten für andere Zwecke nutzen ohne, dass ein Kunde oder Nutzer davon Kenntnis hat. 
Somit ist der allgemeine Datenschutz ein Schutz des Kunden vor genau solch einer Nutzung seiner Daten und bedeutet, dass die Daten eines Kunden oder Nutzers vertraulich behandelt werden um ihn vor missbräuchlicher Datenverarbeitung zu schützen und sein Recht auf informationelle Selbstbestimmung zu gewährleisten. Ebenso kann es als die wahrung seiner Privatsphäre verstanden werden und soll dem Nutzer/Kunden ermöglichen zu entscheiden, wem er wann welche seiner persönlichen Daten zugänglich macht.

\cite{wikiDatenschutz,datenschutz_internet}



\subsection{DSGVO}

\subsection{Intelligente Lernumgebung}

\subsection{ITS}

\subsection{Digitale Lernspiele (Educational Serious Games)}

\section{Stand der Forschung und Technik}

\subsection{Datenschutz \& E-Learning}

\subsection{Datenschutz/DSGVO \& Learning Analytics}

\subsection{Anonymisierungsmethoden}

\subsection{Pseudonymisierung}

Eine weitere Art des Schutzes der zugänglich gemachten Daten eines Nutzers stellt die sogenannte Pseudonymisierung dar. Im Gegensatz zur vollständigen Anonymisierung, welche im vorgehenden Kapitel erläutert wurde, werden hierbei personenbezogene oder persönliche Daten durch das Ersetzen von Kennzeichen oder Pseudonymen verändert. Dies soll eine Identifikation und Bestimmung des Betroffenen ausschließen oder zumindest deutlich erschweren. \cite{thesing_anonym_pseudonym, datenschutz_pseudonymisierung}

\section{Konzeption: E-Learning + Serious Gaming + Learning Analytics + Datenschutz/DSGVO}

\section{Szenario: Learning Analytics für Bildauswertung-Lernspiel}

\section{Fazit \& Ausblick}

\newpage
\listoffigures
\bibliographystyle{unsrt}
\newpage
\bibliography{literatur}

\end{document}